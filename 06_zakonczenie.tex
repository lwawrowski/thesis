W rozprawie podjęto aktualną i ważką tematykę ubóstwa w kontekście estymacji dla małych domen. Przeprowadzone badania empiryczne umożliwiły realizację głównego celu pracy jakim był pomiar ubóstwa na poziomie podregionów i powiatów w Polsce. Przedstawione we wstępie cele szczegółowe również zostały zrealizowane, a także zweryfikowano podstawione w rozprawie hipotezy badawcze. 

W pierwszej kolejności przeanalizowano dorobek prowadzonych w Polsce i na świecie projektów z zakresu ubóstwa oraz estymacji pośredniej. Badania prowadzone przez zespoły międzynarodowe bazowały zwykle na danych symulacyjnych, bądź pochodzących z badania EU-SILC realizowanego we Włoszech lub Hiszpanii. Z kolei polskie projekty badawcze odwoływały się do niewielkiej części potencjału metodycznego i koncentrowały się wyłącznie na estymacji stopy ubóstwa. Niniejsza praca w pewnym stopniu uzupełnia zidentyfikowane luki w zakresie metod oraz badanego zjawiska.

Przeprowadzona analiza źródeł wskazała możliwe drogi do pozyskania zmiennych pomocniczych, niezbędnych w procesie estymacji z wykorzystaniem metod statystyki małych obszarów. Wiele z tych źródeł jest ogólnodostępnych. wśród nich wymienić można Bank Danych Lokalnych oraz inne bazy danych prowadzone przez GUS czy Eurostat. Niektóre z nich, takie jak informacje pozyskiwane przez API, wymagają posiadania pewnej wiedzy informatycznej. Najbogatsze źródło zmiennych pomocniczych, ze względu zarówno na liczbę cech, jak i szczegółowość informacji, stanowią rejestry administracyjne. %Dużym ograniczeniem w ich wykorzystaniu w estymacji pośredniej jest jednak niedostępność danych.

Empirycznie zbadano także jakość danych pochodzących z badania EU-SILC. Wykazano, że znaczna część informacji dotyczących dochodów respondentów jest w nim imputowana. Z~racji tego, że cechy te stanowią podstawę estymacji wskaźników ubóstwa, bardzo ważna jest świadomość pochodzenia tych danych. Przeprowadzona analiza stanowi przyczynek do dalszych badań nad zagadnieniem \textit{informative nonresponse} \citep{laaksonen2006}.

W pracach nad podejściem obszarowym do estymacji stopy oraz głębokości ubóstwa, przeanalizowano wpływ transformacji zmiennej zależnej na jakość oszacowań. Uwagę poświęcono przekształceniu z zastosowaniem pierwiastka arcus sinusa, które w założeniu ma stabilizować wariancję oszacowań bezpośrednich oraz zapewnić wyniki z przedziału $[0;1]$. Badania symulacyjne wykazały, że pomimo pierwotnej poprawy własności rozkładu zmiennej zależnej, oszacowania pośrednie charakteryzowały się większym obciążeniem w porównaniu do modelu, w którym wskaźnik ubóstwa nie był transformowany.

W rozprawie podjęto próbę opracowania modelu wyjaśniającego zmienność stopy oraz głębokości ubóstwa na poziomie podregionów oraz powiatów. Pomimo tego, że oba wskaźniki opisują bardzo zbliżone zjawiska nie było możliwe dopasowanie dobrego modelu z takimi samymi zmiennymi pomocniczymi dla obu cech. Badania empiryczne wykazały, że symptomy wpływające na poziom tych wskaźników pokrywają się tylko częściowo. 

W toku prac ustalono, że najlepszymi własnościami pod względem precyzji, obciążenia oraz poprawności modelu charakteryzuje się model Faya-Herriota z przestrzennie skorelowanym efektem losowym (SEBLUP). Podejście to skutkowało najlepszymi rezultatami w odniesieniu do estymacji wskaźników ubóstwa na poziomie podregionów.

Jakość szacunków otrzymanych na podstawie metod statystyki małych obszarów dla poziomu podregionów skłoniła do podjęcia badań nad wykorzystaniem estymacji pośredniej, także na niższym poziomie agregacji --– lokalnym, reprezentowanym w Polsce przez powiaty. W tym przypadku jednak wyniki otrzymane dzięki estymatorowi SEBLUP charakteryzowały się jedynie nieznaczną poprawą precyzji w odniesieniu do estymacji bezpośredniej. Wynikało to między innymi z występowania wartości odstających w rozkładzie wskaźników ubóstwa.

Mając na uwadze powyższy fakt, w celu zapewnienia rzetelnych oszacowań poziomu ubóstwa na poziomie powiatów zdecydowano się na zastosowanie podejścia jednostkowego. Wykorzystano estymator EB, który w modelu także korzysta z transformowanej wartości zmiennej zależnej --- w tym przypadku dochodu. Przeanalizowano trzy warianty przekształcenia --- logarytm, logarytm z przesunięciem zapewniającym symetryczność rozkładu oraz transformację Boxa-Coxa. Najlepszymi własnościami, w rozumieniu precyzji estymacji charakteryzowały się oszacowania, w których zmienną zależną --- dochód, przekształcono metodą Boxa-Coxa. Oprócz estymatora EB zastosowano także podejście MQ bazujące na regresji kwantylowej. Jest to metoda odporna, która w modelu wykorzystuje oryginalne wartości dochodu. Obie techniki polegają na generowaniu pseudo-populacji metodą Monte Carlo. Otrzymane rezultaty cechowały się dużo lepszą precyzją w odniesieniu do estymacji bezpośredniej. Tym samym potwierdzono hipotezę o lepszej jakości estymacji wykorzystującej zmienne pomocnicze z innych źródeł w porównaniu z podejściem klasycznym.

Z wykorzystaniem metod statystyki małych obszarów dostarczono precyzyjnych oszacowań stopy oraz głębokości ubóstwa na poziomie powiatów. O ile wcześniej podejmowane były próby estymacji stopy ubóstwa w przekroju podregionów i powiatów, tak wartości głębokości ubóstwa były publikowane wyłącznie w wybranych przekrojach społeczno-demograficznych. Uzyskane wyniki istotnie rozszerzają szczegółowość dostępnych informacji, co stanowi novum rozprawy.

Bardzo ważnym zagadnieniem w statystyce małych obszarów jest merytoryczna analiza oszacowań. W przypadku stopy oraz głębokości ubóstwa nie są znane prawdziwe wartości tych cech w populacji. W związku z tym stosuje się podejście polegające na porównaniu wyników estymacji pośredniej do tzw. \textit{zmiennych proxy}. Przyjmuje się, że cechy te mogą stanowić pewne przybliżenie szacowanych parametrów. W niniejszej dysertacji zaproponowano podejście polegające na wykorzystaniu danych dotyczących bezrobocia oraz pomocy społecznej pochodzących z rejestrów administracyjnych w celu oceny oszacowań. Wykazano, że korelacja pomiędzy oszacowaniami pośrednimi, a \textit{zmiennymi proxy} jest wyższa, niż w przypadku oszacowań bezpośrednich. Ponadto, pozytywnie zweryfikowano hipotezę mówiącą o związku pomiędzy poziomem ubóstwa mierzonym stopą i głębokością ubóstwa, a sytuacją na rynku pracy. Relacja ta jest wyraźniejsza w ujęciu lokalnym, na poziomie powiatów, aniżeli w przekroju podregionów.

W pracy zaproponowano także wykorzystanie metod porządkowania liniowego w ocenie oszacowań pośrednich. Na podstawie rankingów skonstruowanych dla wartości stopy oraz głębokości ubóstwa przy wykorzystaniu uogólnionej miary odległości (GDM) weryfikowano zgodność pozycji zajmowanych przez poszczególne domeny. Przeprowadzona analiza wykazała istnienie silnej korelacji pomiędzy modelowymi oszacowaniami wskaźników ubóstwa a rankingiem utworzonym wyłącznie na podstawie zestawu cech diagnostycznych.

Potwierdzono także hipotezę badawczą mówiącą o wyraźnym zróżnicowaniu przestrzennym ubóstwa. Południowo-wschodnia część kraju charakteryzuje się wyższymi wartościami wskaźników ubóstwa w porównaniu do zachodniej części Polski. Zaobserwowano również, że ośrodki miejskie oraz przylegające do nich powiaty są mniej narażone na występowanie zjawiska ubóstwa aniżeli powiaty znacznie oddalone od dużych miast. Na podstawie otrzymanych wyników przeprowadzono także klasyfikacje podregionów i powiatów na grupy charakteryzujące się niskim, przeciętnym oraz wysokim poziomem ubóstwa.

Kartogramy zawarte w pracy, przedstawiające zróżnicowanie poziomu ubóstwa w Polsce, mogą stanowić jedną ze składowych składających się na obraz spójności społecznej Polski. Ponadto taki sposób prezentacji wyników może być formą wsparcia w procesie kształtowania ram polityki społecznej przez wybranych decydentów.

Niniejsza rozprawa doktorska nie wyczerpuje wszystkich zagadnień dotyczących estymacji pośredniej ubóstwa. W dalszych pracach warto byłoby poświęcić uwagę zastosowaniu metod statystyki małych obszarów do estymacji ubóstwa wielowymiarowego \citep{panek2010}. W tym podejściu ubóstwo nie jest postrzegane wyłącznie poprzez pryzmat dochodów czy wydatków, ale także uwzględnia się czynniki pozamonetarne. Do estymacji wskaźników ubóstwa wielowymiarowego można wykorzystać teorię zbiorów rozmytych \citep{panek2009}. Z wykorzystaniem tego podejścia możliwe jest wyznaczenie miar ubóstwa wielowymiarowego dla ogólnych przekrojów terytorialnych np. województw. Niemniej obserwowana jest luka metodyczna dotycząca estymacji tych wskaźników w domenach charakteryzujących się małą liczebnością próby.

Obszarem badawczym, na który także warto zwrócić uwagę jest informatywność schematu losowania oraz braków odpowiedzi \citep{pfeffermann2007}. W przypadku, w którym istnieje zależności pomiędzy szacowanym parametrem a schematem losowania, wyrażonym np. przez wagi przekrojowe, może występować tzw. błąd doboru próby (ang. \textit{sample selection bias}). Takie obciążenie może wpływać na wyniki estymacji, w związku z czym w literaturze proponuje się metody mające na celu uwzględnienie tego czynnika w modelowaniu \citep{verret2015,rao2016}.

Innym kierunkiem, który wymaga zgłębienia w ramach metodyki statystyki małych obszarów jest uwzględnianie w modelu większej liczby efektów losowych --- modele hierarchiczne \citep{jrsssa}. Zastosowanie dodatkowych poziomów w modelu pozwala lepiej opisać zmienność badanego zjawiska. Skutkuje jednak koniecznością przetwarzania bardzo dużych macierzy, co może spowodować trudności w wyznaczeniu optymalnego rozwiązania.

Należy także podnieść problem estymacji pośredniej w okresach, dla których nie są dostępne dane pochodzące ze spisu powszechnego. Ciekawym podejściem związanym z tym zagadnieniem są metody generowania syntetycznych zbiorów danych \citep{Sakshaug2010,roszka2015}. Na podstawie wartości brzegowych pochodzących z badań pełnych, danych z badania reprezentacyjnego oraz odpowiednich modeli wyznaczane są najbardziej prawdopodobne wartości cech na poziomie jednostkowym \citep{lee2009}. Temu zagadnieniu dedykowany jest także pakiet R~o~nazwie \textit{simPop} \citep{simpop2017}.

W szacunku poziomu ubóstwa można także wykorzystać podejście dynamiczne polegające na estymacji parametru z uwzględnieniem korelacji występującej w czasie \citep{raoyu1994}. Oszacowania uzyskane na podstawie danych zgromadzonych przez kilka okresów mogą charakteryzować się lepszą precyzją, aniżeli szacunek dokonany na dany okres \citep{kubacki2016}. Podejście to wymaga dostępu do odpowiednio długiego szeregu czasowego. Oprócz korelacji w czasie modele te mogą także uwzględniać przestrzenną autokorelację efektów losowych \citep{marhuenda2013}.