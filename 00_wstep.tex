Ubóstwo jest uznawane za jeden z najważniejszych i najbardziej złożonych problemów społecznych współczesnego świata. Zjawisko to występuje w każdym społeczeństwie. W niektórych krajach jest ono bardziej widoczne, w innych mniej. Niekoniecznie oznacza to jednak, że różnią się one zasięgiem ubóstwa. Mamy bowiem tu do czynienia z pojęciem zróżnicowanym terytorialnie. Bieda osób mieszkających w Europie znacznie różni się od tej, która dotyka mieszkańców krajów Afryki, czy Azji. Poza wyżej wskazanym wymiarem przestrzennym, ważnym przy określaniu poziomu, czy głębokości ubóstwa jest także wymiar czasowy. Jedną z konsekwencji procesu rozwoju społeczeństwa jest bowiem zmieniający się w czasie status materialny gospodarstwa domowego. Kluczowym problemem zatem w ocenie ubóstwa jest jego pomiar. Nadanie mu uniwersalnego charakteru wymagałoby określenia jednolitych zasad i definicji stanowiących podstawę kryterium wyróżnienia sfery ubóstwa. Jest to jednak bardzo trudne zadanie, którego jak dotąd nie udało się zrealizować.

W przeszłości wielu badaczy podejmowało problem zdefiniowania zjawiska niedostatku. Można w tym miejscu przywołać polskiego ekonomistę, profesora Jana Drewnowskiego, który łączył ubóstwo z niezaspokojeniem potrzeb na oczekiwanym poziomie \citep{drewnowski1977}. Z kolei Amartya Sen, indyjski ekonomista i laureat Nagrody Banku Szwecji im. Alfreda Nobla, wskazał, że ubóstwo to nie tylko niedostępność określonych dóbr i usług, ale także brak możliwości uczestnictwa w podejmowaniu decyzji oraz udziału w życiu społecznym i kulturalnym \citep{sen1992}. Rada Europejskiej Wspólnoty Gospodarczej (WE) w dokumencie z dnia 19 grudnia 1984 roku w sprawie działań mających na celu przeciwdziałanie ubóstwu zdefiniowała to zjawisko w następujący sposób: ,,ubóstwo odnosi się do osób, rodzin lub grup osób, których zasoby (materialne, kulturowe i społeczne) są ograniczone w takim stopniu, że poziom ich życia obniża się poza akceptowalne minimum w kraju zamieszkania'' \citep{eec1985}.

W definicji określonej przez Radę Europejskiej Wspólnoty Gospodarczej kluczowym aspektem jest ustalenie kryterium przynależności do sfery ubóstwa. Wyróżnia się tu dwa wiodące podejścia do zagadnienia: jednowymiarowe lub wielowymiarowe. Stosując kryterium jednowymiarowe ocenia się stopień zaspokojenia potrzeb jednostki poprzez pryzmat dochodów (wydatków) wyrażonych jedynie w formie monetarnej. W literaturze przedmiotu podkreślana jest jednak niedoskonałość tego podejścia oraz przekonanie, że zjawisko ubóstwa nie jest jednowymiarowe \citep{panek2010,jakosc-gus2013}. W~związku z tym wielu badaczy zaczęło postulować o uwzględnienie w jego analizie także czynników pozamonetarnych. Wśród propozycji wymiarów ubóstwa, które miałyby zostać włączone do analizy znalazły się m.in. dochody i zasoby materialne nagromadzone w~poprzednich okresach, warunki mieszkaniowe, edukacja oraz zasoby zawodowe i finansowe \citep{panek2011}. 

Pomimo swoich ograniczeń podejście jednowymiarowe jest wykorzystywane w badaniach prowadzonych przez Główny Urząd Statystyczny (GUS) ze względu na prostotę obliczeń, w których wykorzystuje się wartości wyłącznie jednej cechy --- dochodów bądź wydatków \citep{ubostwo-gus2013}. 

Stosowanie podejścia jednowymiarowego wymaga określenia poziomu dochodów (wydatków), poniżej którego osoba jest uznawana za ubogą. W Polsce zdefiniowanych jest kilka rodzajów granicy (linii) ubóstwa. Instytut Pracy i Spraw Socjalnych (IPiSS) wyznacza wartości: minimum egzystencji (granicę wydatków uwzględniających skromne wyżywienie oraz utrzymanie małego mieszkania), minimum socjalne (granicę wydatków umożliwiającą minimalnie godziwy standard życia), a także bierze udział w konsultacji ustawowej granicy ubóstwa (kwoty dochodu, która uprawnia do ubiegania się przyznanie świadczenia z systemu pomocy społecznej) \citep{kurowski2002}. W badaniach społecznych granicę ubóstwa ustala się jako stałą część mediany lub średniej arytmetycznej rozkładu dochodów w całej populacji. GUS w Badaniu Budżetów Gospodarstw Domowych (BBGD) wykorzystuje linię ubóstwa określoną jako 50\% średnich wydatków. Z kolei Eurostat rekomenduje przyjęcie jako granicy ubóstwa 60\% mediany rozkładu dochodów ekwiwalentnych (uwzględniających skład demograficzny gospodarstwa domowego). 

Pomiar niedostatku bazuje na dwóch podstawowych wskaźnikach: \textit{stopie ubóstwa} oraz \textit{głębokości ubóstwa}. Pierwsza z tych miar informuje o odsetku ubogich w populacji. Wskaźnik ten jest prosty w budowie i interpretacji, niemniej posiada kilka wad. Przede wszystkim, nie bierze pod uwagę intensywności ubóstwa --- osoby znajdujące się poniżej granicy ubóstwa mogą mieć dochody bardzo zbliżone do tej granicy lub bardzo oddalone. W obu przypadkach stopa ubóstwa będzie taka sama, natomiast niewątpliwie większa bieda występuje w drugiej sytuacji. Uzupełnieniem stopy ubóstwa jest głębokość ubóstwa, która informuje o poziomie ubóstwa wśród osób znajdujących się poniżej linii ubóstwa. Oznacza to zatem, że pełniejszy obraz zjawiska otrzymamy prowadząc analizę opartą na obu wymienionych wskaźnikach \citep{haughton2009}.

Badania empiryczne prowadzone w Polsce dostarczają informacji na temat biedy na bardzo ogólnym poziomie. Głównym źródłem oszacowań wskaźników ubóstwa są badania reprezentacyjne prowadzone przez GUS takie jak Badanie Budżetów Gospodarstw Domowych (BBGD) oraz Europejskie Badanie Dochodów i Warunków Życia (EU-SILC). Jak dotąd wartości wskaźników ubóstwa publikowane są na poziomie całego kraju, w przekroju regionów oraz wybranych grup społeczno-demograficznych. Wielkość próby, jak również wykorzystywana obecnie metoda szacunku jaką jest estymacja bezpośrednia nie pozwalają na publikację wyników na niższych poziomach agregacji z powodu wysokich błędów oszacowań. Ponadto z roku na rok coraz więcej osób odmawia udziału w badaniach reprezentacyjnych \citep{dezagregacja2015}. Równolegle do obserwowanych spadków wskaźników realizacji badań reprezentacyjnych rosną potrzeby odbiorców informacji. Oczekuje się danych dla szczegółowych przekrojów oraz coraz mniejszych jednostek administracyjnych czy terytorialnych.

Rozszerzenie pokrycia informacyjnego bez konieczności zwiększenia kosztów badania wynikających między innymi ze zmiany wielkości próby jest możliwe poprzez zastosowanie metod statystyki małych obszarów (estymacji pośredniej). Są to techniki umożliwiające estymację parametrów przy ograniczonej, w skrajnym przypadku zerowej, liczebności próby, wykorzystując w tym celu wszystkie dostępne źródła danych. Ten typ estymacji ma bardzo szerokie spectrum zastosowań. Oprócz badań nad ubóstwem \citep{ell2003,ebp2010,mq2006,esteban2012,pratesi2008}, metody statystyki małych obszarów wykorzystuje się między innymi w szacowaniu charakterystyk rynku pracy \citep{golata2004,wilak2014,molina2007}, niepełnosprawności \citep{elazar2004} oraz w statystyce gospodarczej \citep{dehnel2010,dehnel2017,chandra2012}.

W rozprawie do estymacji wskaźników ubóstwa na poziomie podregionów i powiatów zostaną wykorzystane dwa typy modeli: obszarowe oraz jednostkowe. Bazują one na estymacji danej zmiennej zależnej za pomocą liniowego modelu mieszanego --- zawierającego efekt losowy przyporządkowany zazwyczaj do określonego stopnia podziału administracyjnego \citep{rao2015}. 

Otrzymane wyniki będą mogły stanowić składową obrazu spójności społecznej Polski w ujęciu regionalnym i~lokalnym. Efektem przeprowadzonego badania będą kartogramy na poziomie NUTS 3 i NUTS 4 umożliwiające identyfikację obszarów najbardziej narażonych na występowanie zjawiska ubóstwa. Z kolei wypracowana metodyka może zostać wykorzystana jako narzędzie ewaluacji w prowadzonych obecnie przedsięwzięciach takich jak \textit{Europa 2020}, \textit{Krajowy Program Reform}, \textit{Agenda Post-2015} czy \textit{Krajowy Program Przeciwdziałaniu Ubóstwu i Wykluczeniu Społecznemu 2020. Nowy wymiar aktywnej integracji} mających na celu zmniejszenie poziomu ubóstwa. Wnioski uzyskane na podstawie przeprowadzonych badań będą mogły służyć jako informacja dla władz samorządowych w kontekście planowania efektywnej polityki społecznej.

\subsection*{Cel pracy}

Głównym celem rozprawy jest \textbf{pomiar ubóstwa na poziomie podregionów i powiatów w~Polsce} z wykorzystaniem statystyki małych obszarów. Szacowaniu podlega stopa oraz głębokość ubóstwa w 2011 roku dla wyżej wymienionych przekrojów. 

Osiągnięcie celu głównego będzie możliwe poprzez realizację kolejnych celów szczegółowych:

\begin{itemize}
\item określenie stanu wiedzy dotyczącej pokrycia informacyjnego wskaźników ubóstwa oraz metod estymacji tych wskaźników,
\item charakterystyka i ocena źródeł danych wykorzystywanych w estymacji ubóstwa,
\item analiza potencjalnych źródeł zmiennych pomocniczych, w tym zasobów wykraczających poza tradycyjne bazy danych,
\item adaptacja metod estymacji wskaźników ubóstwa na poziomie podregionów i powiatów w~Polsce wykorzystujących podejście modelowe,
\item statystyczna ocena szacunków pod kątem precyzji i obciążenia,
\item przestrzenna analiza oszacowań wskaźników ubóstwa na poziomie podregionów i powiatów w~Polsce,
\item wyodrębnienie jednostek terytorialnych o podobnych wartościach wskaźników ubóstwa,
\item zastosowanie metod wielowymiarowej analizy statystycznej w ocenie oszacowań wskaźników ubóstwa.
\end{itemize}

\subsection*{Hipotezy badawcze}

W związku z realizacją celów postawionych w rozprawie sformułowano trzy hipotezy badawcze.

\begin{enumerate}
\item Ubóstwo ekonomiczne cechuje się wyraźnym zróżnicowaniem przestrzennym na poziomie lokalnym w Polsce.
\item Estymacja poziomu ubóstwa wykorzystująca zmienne pomocnicze z innych źródeł charakteryzuje się lepszą jakością w porównaniu do estymacji bezpośredniej.
\item Związek pomiędzy poziomem ubóstwa mierzonym stopą i głębokością ubóstwa, a sytuacją na rynku pracy jest wyraźniejszy w ujęciu lokalnym niż w przekroju podregionów.
\end{enumerate}

W polskiej literaturze przedmiotu brakuje kompleksowych analiz poświęconych problematyce estymacji pośredniej wskaźników ubóstwa w ujęciu regionalnym i lokalnym. Ujęcie to zostało podniesione w rozprawie stanowiąc \textbf{novum pracy}. Pomimo wielu opracowań poruszających kwestie ubóstwa, rzadkie są prace dotyczące estymacji zarówno stopy, jak i głębokości ubóstwa. Za oryginalny wkład rozprawy uznać można w pierwszym rzędzie adaptację metod pośrednich wykorzystywanych w badaniu ubóstwa na świecie do ram polskiego systemu statystyki publicznej. Oryginalnym wkładem jest również przedstawienie terytorialnego zróżnicowania głębokości ubóstwa w przekroju podregionów i powiatów w oparciu o przeprowadzoną estymację. Na podkreślenie zasługuje również zawarta w pracy ocena uzyskanych oszacowań, która ma charakter wielopłaszczyznowy. Wykorzystano w niej dostępne informacje pochodzące z rejestrów administracyjnych oraz metody wielowymiarowej analizy statystycznej.

\subsection*{Charakterystyka zastosowanych metod badawczych}

W pracy podjęto próbę estymacji wskaźników ubóstwa na poziomie podregionów i powiatów z~wykorzystaniem modeli statystyki małych obszarów. Jako zmienną zależną przyjęto wskaźnik ubóstwa mierzony na określonym poziomie terytorialnym (w podejściu obszarowym) lub dochód gospodarstwa (w podejściu jednostkowym). Po drugiej stronie równania jako zmienne niezależne przyjęto tzw. \textit{zmienne pomocnicze} --- informacje silnie skorelowane ze zmienną zależną pochodzące z dodatkowego źródła danych.

W badaniu kluczową rolę przy budowie modelu wykorzystanego w estymacji, odegrała identyfikacja symptomów ubóstwa. Pełniły one w procesie modelowania rolę informacji pomocniczej. Wskazanie cech stanowiących determinanty ubóstwa pozwoliło na stworzenie bazy tzw. \textit{zmiennych pomocniczych}. Wśród najistotniejszych symptomów ubóstwa znalazły się takie cechy jak: źródło utrzymania gospodarstwa, liczebność gospodarstwa, wykształcenie członków gospodarstwa domowego czy występowanie w nim osób bezrobotnych bądź niepełnosprawnych.

W modelach obszarowych poziom analizowanego wskaźnika ubóstwa w danym obszarze objaśniany jest zmiennymi mierzonymi na poziomie tego obszaru. Wykorzystuje się w tym celu charakterystyki społeczno-ekonomiczne dostępne w powszechnych bazach statystycznych takich jak np. Bank Danych Lokalnych. Przykładowo poziom stopy ubóstwa w danym powiecie może być wyjaśniony przez odsetek osób nieposiadających wyższego wykształcenia w tym powiecie. Możliwość wykorzystania bogatszego zbioru zmiennych pomocniczych, w porównaniu z podejściem jednostkowym, stanowi bardzo ważną zaletę modeli obszarowych. Najpopularniejszym przedstawicielem tej grupy metod jest model Faya-Herriota opracowany w celu szacowania średniego dochodu w małych domenach w Stanach Zjednoczonych \citep{fh1979}. Podejście to jest cały czas rozwijane m.in. poprzez ujęcie w modelu macierzy sąsiedztwa i uwzględnienie w ten sposób czynnika przestrzennego \citep{pratesi2008}. To rozszerzenie charakteryzuje się lepszymi własnościami, niż klasyczny model Faya-Herriota, ale wymaga spełnienia dodatkowych założeń, takich jak występowania istotnej korelacji przestrzennej.

Modele jednostkowe z kolei bazują na modelowaniu dochodów lub wydatków gospodarstw domowych z wykorzystaniem danych jednostkowych pochodzących z badań pełnych lub rejestrów administracyjnych. W tym przypadku dostępność zmiennych niezależnych jest dużo mniejsza od tej, występującej dla modeli obszarowych z powodów wynikających między innymi z~zachowania tajemnicy statystycznej. Dochód gospodarstwa domowego może być objaśniany przez charakterystyki tego gospodarstwa takie jak: liczba osób bezrobotnych czy wskaźnik obciążenia demograficznego. Do najważniejszych technik estymacji opartych na modelach jednostkowych należą metody ELL \citep{ell2003}, M-kwantylowa (MQ) \citep{mq2006} oraz empiryczna bayesowska (EB) \citep{ebp2010}. Estymacja danego wskaźnika ubóstwa w przypadku tych metod polega na tworzeniu, z~wykorzystaniem symulacji Monte Carlo, pseudo-populacji, które stanowią podstawę estymacji wskaźników ubóstwa. Wymienione metody są do siebie bardzo podobne jeśli chodzi o ideę. Dzielą je jednak istotne różnice. Metoda ELL nie uwzględnia bowiem w ogóle danych pochodzących z badania reprezentacyjnego w przeciwieństwie do metody EB. Z~kolei w metodzie MQ liniowy model mieszany został zastąpiony przez model regresji kwantylowej.

W dysertacji zastosowano wyżej przedstawione modele do estymacji stopy oraz głębokości ubóstwa na poziomie podregionów i powiatów. Rozważano także różne metody transformacji zmiennej zależnej, które miały na celu poprawę własności predykcyjnych modeli \citep{analpovdata52016}. Obciążenie oraz precyzja wszystkich otrzymanych szacunków zostały ocenione za pomocą metody bootstrap \citep{gonzales2008}. Na tej podstawie dokonano wyboru najlepszej metody estymacji w odniesieniu do danego poziomu terytorialnego.

Wszystkie obliczenia oraz wizualizacje zostały wykonane w pakiecie R \citep{r2016}.

\subsection*{Źródła danych}

Do estymacji poziomu ubóstwa w Polsce zastosowano dane jednostkowe z Europejskiego Badania Dochodów i Warunków Życia (EU-SILC) z roku 2011. Wybór tego okresu analizy podyktowany został dostępnością danych statystycznych wykorzystanych w estymacji pośredniej w charakterze \textit{zmiennych pomocniczych}. Ich źródło stanowiły informacje zebrane w ramach Narodowego Spisu Powszechnego Ludności i Mieszkań przeprowadzonego w 2011 roku. Oprócz tego wspomagano się danymi pochodzącymi z Banku Danych Lokalnych. 

\subsection*{Treść pracy}

Praca ma charakter teoretyczno-empiryczny i składa się ze wstępu, pięciu rozdziałów, zakończenia, bibliografii oraz załączników. Pierwsze trzy rozdziały mają charakter teoretyczny, a ich podstawą są badania literaturowe. Z kolei czwarty i piąty rozdział zawiera wyniki przeprowadzonych badań empirycznych --- oszacowania stopy oraz głębokości ubóstwa na poziomie podregionów i~powiatów wraz z ich statystyczną oraz merytoryczną oceną.

W \textbf{rozdziale pierwszym} zostały przedstawione teoretyczne aspekty pomiaru ubóstwa. Na podstawie literatury przedmiotu zidentyfikowano najważniejsze determinanty ubóstwa. W treści rozdziału poświęcono także miejsce na przegląd projektów badawczych dotyczących statystycznej analizy ubóstwa. Duża liczba takich inicjatyw świadczy o aktualności problemu badawczego. Ponadto w rozdziale, na podstawie dostępnych źródeł przedstawiono czasowy oraz przestrzenny wymiar ubóstwa w Polsce. 

\textbf{Rozdział drugi} zawiera opis źródeł danych wykorzystywanych w badaniu ubóstwa. Uwagę skupiono w pierwszej kolejności na źródłach, na których obecnie oparte są publikacje dotyczące niedostatku. Scharakteryzowano wybrane badania reprezentacyjne: Badanie Budżetów Gospodarstw Domowych, Europejskie Badanie Dochodów i Warunków Życia, Badanie Spójności Społecznej oraz Diagnozę Społeczną. W dalszej kolejności skupiono się na spisach ludności i~rejestrach administracyjnych, które stanowią odpowiednie źródło zmiennych pomocniczych w estymacji pośredniej. W rozdziale przedstawiono także tzw. \textit{niestatystyczne źródła danych}. W~dobie powszechnego dostępu do Internetu wskazuje się, że do szacowania ubóstwa można wykorzystać dane z~serwisów społecznościowych, a źródłem zmiennych pomocniczych mogą być chociażby serwisy z~mapami. Możliwe jest także wykorzystanie nocnych zdjęć satelitarnych. 

Opis zastosowanych metod znajduje się w \textbf{rozdziale trzecim}. W pierwszej kolejności scharakteryzowano estymację bezpośrednią, stanowiącą podstawową technikę szacowania parametrów populacji. Następnie przedstawiono wybrane modele estymowane na poziomie obszaru: klasyczny model Faya-Herriota oraz jego wariant uwzględniający przestrzenne skorelowanie efektów losowych. Spośród modeli estymowanych na poziomie jednostki szczegółowo scharakteryzowano metodę empiryczną bayesowską (EB) oraz M-kwantylową (MQ). W ostatniej części rozdziału zaprezentowano metody wykorzystane do diagnostyki otrzymanych wyników pod kątem m.in. obciążenia i precyzji szacunku.

\textbf{Rozdział czwarty} jest podzielony na dwie części. W pierwszej dokonano oceny jakości danych jednostkowych pochodzących z Europejskiego Badania Dochodów i Warunków Życia 2011. W analizie skupiono się na badaniu wpływu sposobu udzielania odpowiedzi oraz imputacji na wartości dochodu respondentów. Druga część dotyczy estymacji dwóch wskaźników --- stopy oraz głębokości ubóstwa na poziomie podregionów i powiatów. W tym celu zidentyfikowano odpowiednie symptomy ubóstwa, co dało podstawy do empirycznego zastosowania metod przedstawionych w rozdziale trzecim. Rozważano także wpływ transformacji estymowanej cechy na jakość oszacowań. W modelach na poziomie obszaru zastosowano przekształcenie za pomocą pierwiastka arcus sinusa. Natomiast w modelach na poziomie jednostki wykorzystano transformację logarytmiczną, logarytmiczną z przesunięciem oraz Boxa-Coxa. Oszacowania wskaźników ubóstwa poddano ocenie pod kątem precyzji oraz obciążenia. Zweryfikowane zostały również założenia dotyczące rozkładów reszt i efektów losowych w zastosowanych metodach.

W \textbf{rozdziale piątym} dokonano porównania otrzymanych oszacowań z dostępnymi danymi pochodzącymi z rejestrów administracyjnych. Dzięki temu merytorycznie oceniono stopę oraz głębokość ubóstwa otrzymaną z wykorzystaniem metod statystyki małych obszarów. Wykorzystano także metody porządkowania liniowego w celu wielopłaszczyznowej oceny oszacowań. Otrzymane wskaźniki ubóstwa przedstawiono także na kartogramach, dzięki czemu możliwa była identyfikacja enklaw biedy występujących w Polsce. 

W \textbf{zakończeniu} rozprawy podsumowano uzyskane wyniki badania dotyczące estymacji wskaźników ubóstwa przedstawione w kolejnych rozdziałach pracy. Wykazano, że zastosowanie metod statystyki małych obszarów umożliwia uzyskanie wiarygodnych i precyzyjnych oszacowań stopy oraz głębokości ubóstwa na tych poziomach agregacji przestrzennej, dla których dane obecnie nie są publikowane. Zaproponowana w rozprawie procedura badawcza umożliwia wyjście naprzeciw zgłaszanemu przez odbiorców danych zapotrzebowaniu informacyjnemu, a uzyskane wyniki mogą zostać wykorzystane między innymi przez władze samorządowe do oceny oraz planowania polityki społecznej.